\ssect{Введение}\label{chap:introduction}

Антенные решётки играют важную роль во многих сферах современной жизни. Их применение можно увидеть, например, в следующих областях:

\begin{itemize}
      \item Системы радиосвязи
            \begin{itemize}
                  \item В базовых станциях мобильной связи для создания узких областей покрытия, что помогает
                        обеспечивать связь в труднодоступных зонах и осуществлять пространственное
                        разделение в густонаселенных районах, что позволяет увеличить пропускную способность
                  \item В системах спутниковой связи для осуществления покрытия сигналом больших областей
                        и улучшения качества связи за счёт создания узких, но быстро перестраиваемых диаграмм направленности.
            \end{itemize}
      \item Радиолокационные системы
            \begin{itemize}
                  \item Применение антенных решёток в военных, авиационных и автомобильных радарах позволяет быстро сканировать пространство
                        и эффективно обнаруживать объекты и вычислять их параметры (местоположение, скорость и направление движения);
                        осуществлять слежение за быстро передвигающимися целями и принимать решения на основе полученных данных.
            \end{itemize}
      \item Радиоастрономия
            \begin{itemize}
                  \item В радиотелескопах, таких как Very Large Array (VLA), для изучения космических
                        радиоволн, исходящих от различных астрономических объектов.
            \end{itemize}
\end{itemize}

Несмотря на свои многочисленные достоинства, активные фазированные антенные решётки обладают такими недостатками
как сложность разработки, возможные большие размеры и высокая стоимость. В зависимости от области применения инженерам
приходится находить оптимальные соотношения между сложностью, стоимостью, скоростью сканирования, размерами, ремонтопригодностью
и многими другими параметрами антенных решёток, которые противоречат друг другу. Ввиду большого количества учитываемых параметров,
многокритериальная оптимизация требует комплексного подхода и глубокого понимания взаимодействия между ними.


Цель данной магистерской работы заключается в проведении обзора существующих методов оптимизации антенных решёток с последующим анализом и сравнением их эффективности.

Задачами данной работы являются:

\begin{itemize}
      \item Исследовать различные геометрические и амплитудно-фазовые распределения
      \item Исследовать метод итеративной обработки данных
      \item Исследовать возможности применения интерполяции в антенных решётках
      \item Исследовать применения технологии MIMO в радиолокаторах
\end{itemize}

Работа направлена на исследование того, как различные методы оптимизации могут влиять на важные параметры системы - такие как сложность, стоимость,
размер и скорость сканирования. На основе проведённого анализа будет сформулирован ряд рекомендаций по выбору и
применению методов в зависимости от специфики задачи и условий эксплуатации.