\ssect{Введение}\label{chap:introduction}

Антенные решетки играют важную роль во многих современных приложениях - от мобильной связи, до спутниковой,
в телевещании, в радиолокации, в радиоастрономии и в других областях беспроводных технологий.
Одними из самых важных областей их использования являются:

\begin{itemize}
    \item Системы радиосвязи
        \begin{itemize}
            \item В базовых станциях мобильной связи для управления направлением и мощностью сигнала, что помогает
            увеличить пропускную способность и уменьшить помехи.
            \item В системах спутниковой связи для передачи и приема сигналов с высокой точностью и эффективностью.
            \item В современных технологиях Wi-Fi и 5G для управления передачей данных в беспроводных сетях,
            улучшения покрытия и уменьшения зон с плохим сигналом.
        \end{itemize}
    \item Радиолокационные системы
    \begin{itemize}
        \item В военных и гражданских радарных системах для точного определения положения,
        скорости и направления объектов. Это включает в себя авиационные радары, морские радары
        и радары контроля дорожного движения.
        \item В автомобильных радарах для систем помощи водителю, таких как адаптивный круиз-контроль
        и системы предотвращения столкновений.
    \end{itemize}
    \item Радиоастрономия
    \begin{itemize}
        \item В радиотелескопах, таких как Very Large Array (VLA), для изучения космических радиоволн,
        исходящих от различных астрономических объектов.
    \end{itemize}
  \end{itemize}

Применение антенных решёток в этих областях позволяет значительно увеличить их эффективность за счёт возможности
управления направлением сигнала и концентрации его мощности в данном направлении.

Однако, несмотря на свои многочисленные достоинства, они имеют такие недостатки как сложность, возможные
большие размеры и дороговизна. В зависимости от области применения инженерам приходится находить оптимальные
соотношения между сложностью, стоимостью таких систем, скоростью их работы, размерами, ремонтопригодностью и
многими другими параметрами. При этом одни параметры зачастую (всегда) противоречат друг другу, и потому
многокритериальная оптимизация антенных систем требует комплексного подхода и глубокого понимания взаимодействия
между различными параметрами.

Цель данной магистерской работы заключается в проведении обзора существующих методов оптимизации антенных решёток
с последующим анализом и сравнением их эффективности.

Задачами данной работы являются:

\begin{itemize}
    \item Исследование различных геометрических и амплитудных распределений
    \item Исследование возможностей применения метаповерхностей в сканирующих антенных решётках
    \item Исследовать метод многократного применения одних и тех же данных в различных комбинациях для
    улучшения диаграммообразования
    \item Исследовать возможность применения интерполяции в антенных решётках
    \item Исследовать применение технологии MIMO
\end{itemize}

Работа направлена на исследование того, как различные методы оптимизации могут влиять на критически важные параметры
системы, такие как цена, размер, сложность и скорость работы. Основываясь на анализе, будет сформулирован ряд
рекомендаций по выбору и применению методов оптимизации в зависимости от специфики задачи и условий эксплуатации.