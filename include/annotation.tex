{\Large\normalfont\bfseries \hfil Аннотация}
\vspace{1em}
\thispagestyle{empty}

Магистерская диссертация на тему <<методы уменьшения количества элементов в современных приёмных антенных решётках>>.

Работа состоит из введения, 6 глав, заключения, списка источников и 2 приложений.

Во введении раскрывается актуальность данной работы и описываются поставленные задачи.

В первой главе рассматриваются классические антенные решётки с равномерным расположением элементов. Показаны недостатки данных систем

Во второй главе рассматриваются классические методы уменьшения количества элементов~--~неравномерное распределение
элементов и амплитуд. Показываются и сравниваются распространённые схемы.

В третьей главе приводится краткое введение в цифровые антенные решётки (ЦАР).

В четвёртой главе описан итеративный метод радиолокации.
Проводится моделирование.
Описаны достоинства и недостатки метода, возможности для дальнейшего развития.

В пятой главе описаны возможности применения интерполяции в радиолокации с использованием ЦАР. 
Проводится моделирование. Описаны недостатки и предполагаемые достоинства метода,
приводятся направления для дальнейших исследований.

В шестой главе рассказно про возможности MIMO радиолокаторов. Проведено моделирование когерентного радиолокатора. 
Описаны достоинства и недостатки MIMO радаров. 

Работа содержит~\pageref*{LastPage} страниц, \totaltables~таблиц, \totalfigures~рисунок.
Для написания данной работы было использовано 28 источников.


\newpage
{\Large\normalfont\bfseries \hfil Annotation}
\vspace{1em}
\thispagestyle{empty}

Master's thesis on the topic <<Methods for Reducing the Number of Elements in Modern Receiving Antenna Arrays>>.

The work consists of an introduction, 6 chapters, a conclusion, a list of sources, and 2 appendices.

The introduction discusses the relevance of this work and describes the tasks set.

The first chapter examines classical antenna arrays with uniform element placement.
The disadvantages of these systems are shown.

The second chapter discusses classical methods for reducing the number of elements—unequal
distribution of elements and amplitudes. Common schemes are shown and compared.

The third chapter provides a brief introduction to digital antenna arrays (DAAs).

The fourth chapter describes an iterative method of radar detection. Simulation is conducted.
The advantages and disadvantages of the method are described, with possibilities for further development.

The fifth chapter describes the possibilities of using interpolation in radar with the use of DAAs.
Simulation is conducted. The disadvantages and assumed advantages of the method are described, along
with directions for further research.

The sixth chapter discusses the capabilities of MIMO radars. Simulation of a coherent radar is conducted.
The advantages and disadvantages of MIMO radars are described.

The work contains~\pageref*{LastPage} pages, \totaltables~tables, \totalfigures~figures. 
A total of 28 sources were used to write this work.
