\subsection{Цифровые антенные решётки}\label{sect:daa}

Цифровые антенные решётки (ЦАР) представляют собой передовую технологию в области антенных систем. ЦАР позволяют 
управлять антенной на цифровом уровне, что обеспечивает большую гибкость гибкость и точность формирования ДН, а также 
синтеза многолучевых систем.

Цифровые антенные решётки основаны на тех же принципах что и классические АФАР, однако отличаются по своей архитектуре. 
Каждый канал в ЦАР представляет собой отдельный цифровой радиоприёмник. Принятые данные обрабатываются в цифровой 
подсистеме на ПЛИС и высокопроизводительных процессорах. 

Преимущества цифровых антенных решёток:

\begin{enumerate}
    \item Большая гибкость в формировании диаграммы направленности
    \item Эффективное формирование многолучевых ДН. Так как вся обработка происходит в цифровом виде, 
    то можно формировать сложные сигналы без потерь на сумматорах/делителях мощности. Особенно в приёмных ЦАР
    \item Увеличение скорости сканирования - различные алгоритмы цифровой обработки сигналов 
    (например алгоритм быстрого преобразования Фурье) позволяют значительно ускорить построение 
    картины сканирования пространства
\end{enumerate}

Проблемы и недостатки ЦАР:

\begin{enumerate}
    \item Сложность разработки - ЦАР представляют собой более сложные, многоканальные радиотехнические устройства 
    со своими особенностями.
    \item Повышенные требования к теплоотводу - высокая интеграция цифровых и аналоговых микросхем повышает 
    требования к теплоотводу, так как на меньшей площади теперь выделяется больше энергии.
    \item Увеличение сложности оборудования обработки данных - ЦАР генерирует большой поток данных, 
    который необходимо успевать обрабатывать
\end{enumerate}

Следующие разделы описывают методы оптимизации с использованием цифровых антенных решёток.