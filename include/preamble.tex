%%% Работа с русским языком
\usepackage[english,russian]{babel}   %% загружает пакет многоязыковой вёрстки
\usepackage{fontspec}      %% подготавливает загрузку шрифтов Open Type, True Type и др.
\defaultfontfeatures{Ligatures={TeX},Renderer=Basic}  %% свойства шрифтов по умолчанию
\setmainfont[Ligatures={TeX,Historic}]{Times New Roman} %% задаёт основной шрифт документа
\setsansfont{Arial}                    %% задаёт шрифт без засечек
\setmonofont{Courier New}
\usepackage{indentfirst}
\frenchspacing

%%% Дополнительная работа с математикой
\usepackage{amsmath,amsfonts,amssymb,amsthm,mathtools} % AMS
\usepackage{icomma} % "Умная" запятая: $0,2$ --- число, $0, 2$ --- перечисление

\renewcommand{\epsilon}{\ensuremath{\varepsilon}}
\renewcommand{\phi}{\ensuremath{\varphi}}
\renewcommand{\kappa}{\ensuremath{\varkappa}}
\renewcommand{\le}{\ensuremath{\leqslant}}
\renewcommand{\leq}{\ensuremath{\leqslant}}
\renewcommand{\ge}{\ensuremath{\geqslant}}
\renewcommand{\geq}{\ensuremath{\geqslant}}
\renewcommand{\emptyset}{\varnothing}

%%% Работа с картинками
\usepackage{graphicx}  % Для вставки рисунков
\graphicspath{%
	{images/equally-spaced-arrays},
	{images/array-distributions-theory},
	{images/array-distributions-modeling},
	{images/interpolation},
	{images/iterative},
	{images/mimo},
}  % папки с картинками
%\usepackage{subfig}
\setlength\fboxsep{3pt} % Отступ рамки \fbox{} от рисунка
\setlength\fboxrule{1pt} % Толщина линий рамки \fbox{}
\usepackage{wrapfig} % Обтекание рисунков текстом
\usepackage[lflt]{floatflt}
\usepackage[labelfont=bf]{caption}
\usepackage{subcaption}
\renewcommand\thesubfigure{\asbuk{subfigure}}
\counterwithin{figure}{section}

\DeclareCaptionLabelFormat{gostfigure}{Рисунок #2}
\DeclareCaptionLabelFormat{gosttable}{Таблица #2}
\DeclareCaptionLabelSeparator{gost}{~--~}
\captionsetup{labelsep=gost}
\captionsetup[figure]{labelformat=gostfigure}
\captionsetup[table]{labelformat=gosttable}

\usepackage{float}
% Работа с графиками
\usepackage{tikz}
\usepackage{pgfplots}
\usepackage{pgfplotstable}

%%% Работа с таблицами
\usepackage{array,tabularx,tabulary,booktabs} % Дополнительная работа с таблицами
\counterwithin{table}{section}
%%%%%%%%%%%%%%%%%%%%%%%%%%%%%%%%%%%%%%%%%%%%%%%%%%%%%%%%%%%%%%%%%%%%%%%%
\usepackage{tabularray}
\DefTblrTemplate{caption-tag}{default}{\textbf{\tablename\hspace{0.25em}\thetable}}
\DefTblrTemplate{caption-sep}{default}{\enskip}
\DefTblrTemplate{caption-text}{default}{\InsertTblrText{caption}}

\DefTblrTemplate{caption}{default}{
	\raggedleft
	\UseTblrTemplate{caption-tag}{default}
	\UseTblrTemplate{caption-sep}{default}
	\UseTblrTemplate{caption-text}{default}

}

\DefTblrTemplate{contfoot-text}{default}{Продолжение на следующей странице}
\SetTblrTemplate{contfoot-text}{default}

\DefTblrTemplate{conthead-text}{default}{(Продолжение)}

\DefTblrTemplate{conthead}{default}{
	\UseTblrTemplate{conthead-text}{default}
}

\DefTblrTemplate{capcont}{default}{
	\hfill
	\UseTblrTemplate{caption-tag}{default}
	\UseTblrTemplate{caption-sep}{default}
	\UseTblrTemplate{caption-text}{default}
	\UseTblrTemplate{conthead-text}{default}
}
%%%%%%%%%%%%%%%%%%%%%%%%%%%%%%%%%%%%%%%%%%%%%%%%%%%%%%%
\usepackage{multirow} % Слияние строк в таблице
\usepackage{array}

%%% Программирование
\usepackage{tcolorbox}
\usepackage{etoolbox} % логические операторы

%%% Страница
\usepackage{extsizes} % Возможность сделать 14-й шрифт
\usepackage{geometry}
\geometry{top=20mm}
\geometry{bottom=20mm}
\geometry{left=30mm}
\geometry{right=15mm}
%%%
\righthyphenmin=2

%%% Настройка отображения содержания
\usepackage{tocloft}
\renewcommand{\cftpartleader}{\cftdotfill{\cftdotsep}} % for parts
\renewcommand{\cftsecleader}{\cftdotfill{\cftdotsep}} % for sections
%%% Заголовки
\usepackage{titlesec}
\titleformat{\section}{\Large\normalfont\bfseries\raggedright}{\arabic{section}.}{0.5ex}{}
\titleformat{\subsection}{\large\normalfont\bfseries\raggedright}{\arabic{section}.\arabic{subsection}.}{0.5ex}{}
\titleformat{\subsubsection}{\normalfont\bfseries\raggedright}{\arabic{section}.\arabic{subsection}.\arabic{subsubsection}.}{0.5ex}{}
\titlespacing*{\section}{0pt}{\baselineskip}{\baselineskip}

\providecommand\phantomsection{}
%\newcommand\schap[1]{\chapter*{#1} \addcontentsline{toc}{chapter}{#1} \phantomsection}
\newcommand\ssect[1]{\section*{#1} \phantomsection \addcontentsline{toc}{section}{#1} \markboth{#1}{#1}}
\newcommand{\rwn}{\the\numexpr\value{rownum}-1\relax}
%%% Custom appendix naming
\usepackage[toc,page]{appendix}
\renewcommand\appendixname{Приложение}
\renewcommand\appendixpagename{Приложения}
\renewcommand\appendixtocname{Приложения}



%%% Ссылки
%\usepackage[superscript]{cite} % Ссылки в верхних индексах
%\usepackage[nocompress]{cite} % 
\usepackage{csquotes} % Еще инструменты для ссылок
%\renewcommand{\refname}{Список источников}
%\addto\captionsrussian{\def\refname{Список источников}}

\usepackage[%
	backend=biber,% движок
	bibencoding=utf8,% кодировка bib файла
	sorting=none,% настройка сортировки списка литературы
	style=gost-numeric,% стиль цитирования и библиографии (по ГОСТ)
	language=autobib,% получение языка из babel/polyglossia, default: autobib % если ставить autocite или auto, то цитаты в тексте с указанием страницы, получат указание страницы на языке оригинала
	autolang=other,% многоязычная библиография
	clearlang=true,% внутренний сброс поля language, если он совпадает с языком из babel/polyglossia
	defernumbers=true,% нумерация проставляется после двух компиляций, зато позволяет выцеплять библиографию по ключевым словам и нумеровать не из большего списка
	sortcites=true,% сортировать номера затекстовых ссылок при цитировании (если в квадратных скобках несколько ссылок, то отображаться будут отсортированно, а не абы как)
	doi=false,% Показывать или нет ссылки на DOI
	isbn=false,% Показывать или нет ISBN, ISSN, ISRN
	mincitenames=3,
	maxcitenames=3,
	maxbibnames=3,
]{biblatex}

\addbibresource{include/references.bib}

\DefineBibliographyStrings{russian}{%
	bibliography = {Список источников},
}

%\renewcommand{\baselinestretch}{1.15}
%\usepackage{setspace} % Интерлиньяж
%\spacing{1.15}
%\onehalfspacing % Интерлиньяж 1.5
%\doublespacing % Интерлиньяж 2
%\singlespacing % Интерлиньяж 1

\usepackage{lastpage} % Узнать, сколько всего страниц в документе.


\usepackage{soul} % Модификаторы начертания
\usepackage{hyperref}
\hypersetup{%
	colorlinks=true,
	linkcolor=blue,
	filecolor=magenta,
	urlcolor=cyan,
	pdfauthor={Лазба Филипп},
	pdftitle={Магистерская диссертация на тему «исследование методов оптимизации современных антенных решеток»},
	pdfsubject={Исследование методов оптимизации современных антенных решеток},
	pdfkeywords={АФАР; ЦАР; AESA; DAA; антенные решётки},
	bookmarks=true,
	pdfpagemode=UseThumbs,
}


%\renewcommand{\familydefault}{\sfdefault} % Начертание шрифта

\usepackage{multicol} % Несколько колонок




\renewcommand{\epsilon}{\ensuremath{\varepsilon}}
\renewcommand{\phi}{\ensuremath{\varphi}}
\renewcommand{\kappa}{\ensuremath{\varkappa}}
\renewcommand{\le}{\ensuremath{\leqslant}}
\renewcommand{\leq}{\ensuremath{\leqslant}}
\renewcommand{\ge}{\ensuremath{\geqslant}}
\renewcommand{\geq}{\ensuremath{\geqslant}}
\renewcommand{\emptyset}{\varnothing}

\newcommand{\bhline}[1]{\noalign{\hrule height #1pt}}



\renewcommand{\theenumii}{\arabic{enumi}.}
\renewcommand{\theenumii}{\arabic{enumi}.\arabic{enumii}}
\renewcommand{\theenumiii}{\arabic{enumi}.\arabic{enumii}.\arabic{enumiii}}

% \usepackage{xassoccnt}
% \NewTotalDocumentCounter{totalfigures}
% \NewTotalDocumentCounter{totaltables}
% \DeclareAssociatedCounters{figure}{totalfigures}
% \DeclareAssociatedCounters{table}{totaltables}

% \usepackage{listings}
% \usepackage{matlab-prettifier}

\usepackage{minted}
\usemintedstyle{friendly_grayscale}
\renewcommand\theFancyVerbLine{\normalsize\arabic{FancyVerbLine}}

%%%%%%%%%%%%%%%%%%%%%%%%%%%%%%%%%%%%%%%%%%%%%%%%%%%%%%%%%%%%%%%

\newcommand*{\figoneendian}{\totalfigures~рисунок}
\newcommand*{\figfourendian}{\totalfigures~рисунка}
\newcommand*{\figfiveendian}{\totalfigures~рисунков}

% Расчёт значений для общего числа рисунков
\newcommand*{\printtotfig}[1][,]{%
	\newbool{less11Q}%
	\newbool{greater19Q}%
	\boolfalse{less11Q}%
	\boolfalse{greater19Q}%
	\newcounter{Ptempnumexpr}%
	\newcounter{Pwordindex}%
	\setcounter{Ptempnumexpr}{0}%
	\addtocounter{Ptempnumexpr}{\numexpr\totalfigures-100*(\totalfigures/100)}%
	\ifnumequal{\totalfigures}{-1}%
	{??#1}%
	{\ifnumequal{\totalfigures}{0}{}%
		{\ifnumless{\value{Ptempnumexpr}}{11}{\booltrue{less11Q}}{}%
			\ifnumgreater{\value{Ptempnumexpr}}{19}{\booltrue{greater19Q}}{}%
			\ifboolexpr{bool {less11Q} or bool {greater19Q}}%
			{\setcounter{Pwordindex}{0}%
				\addtocounter{Pwordindex}{\numexpr\value{Ptempnumexpr}-10*(\value{Ptempnumexpr}/10)}%
				\ifnumequal{\value{Pwordindex}}{1}% Окончание для 1
				{\figoneendian}%
				{\ifnumequal{\value{Pwordindex}}{2}% Окончание для 4
					{\figfourendian}%
					{\ifnumequal{\value{Pwordindex}}{3}%
						{\figfourendian#1}%
						{\ifnumequal{\value{Pwordindex}}{4}%
							{\figfourendian}%
							{\figfiveendian}% Окончание для 5
						}%
					}%
				}%
			}%
			{\figfiveendian}% Окончане для 5
		}%
	}%
}