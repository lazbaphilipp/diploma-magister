\ssect{Итоги исследования}\label{chap:conclusions}

В данной магистерской диссертации были рассмотрены различные аспекты проектирования и функционирования современных 
антенных решёток и радиолокационных систем. Исследование охватило несколько тем, начиная от методов 
применимым к классическим антенным решёткам - неравномерные амплитудные распределения, 
проектирование неэквидистантных антенных решёток; 
и заканчивая рядом методов применимым к современным цифровым антенным решёткам. 

В ходе выполнения работы были решены следующие задачи

\begin{enumerate}
    \item Рассмотрены методы формирования неравномерных амплитудных распределений, 
    проведено моделирование и сравнение результатов
    \item Изучены методы проектирования антенных решёток с непостоянным межэлементным расстоянием. 
    Проведено их сравнение, описаны возможности и особенности их эффективного применения для различных задач. 
    Проведено моделирование и сравнение результатов. Сделаны выводы по разработанным моделям.
    \item Изучен метод итеративной обработки данных при диаграммообразовании в цифровых антенных решётках. 
    Приведены достоинства и недостатки метода. 
    Указаны возможности для дальнейших исследований и пути развития.
    \item Рассмотрено применение MIMO радиолокаторов. 
    Проведено моделирование когерентного MIMO радара. 
    Изучены практические и теоретические аспекты их применения в радиолокации. 
    Приведены достоинства и недостатки. 
    Сделаны выводы по возможностям их использования в современных системах.
\end{enumerate}

Результатом данной работы является разработанный набор моделей, 
которые могут быть использованы в дальнейших исследованиях и в рабочих процессах 
при выборе типа и проектировании антенных решёток. 


